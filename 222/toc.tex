% !TEX TS-program = pdftex


\def\chapter#1#2{

{\bf Chapter #1 -- #2} }
\def\section#1#2{

\quad\S #1 --#2 }
\def\subsection#1#2{}
%{
%
%\quad\quad\S {\it#1 --#2}}
\def\subsubsection#1#2{}
%{
%
%\quad\quad\quad\S {\it #1 --#2}}
\def\ref#1{(--)}


\chapter{I}{Methods of Integration}
\section{1}{Definite and indefinite integrals}
\subsection{1.1}{Definition}

\section{3}{First trick: using the double angle formulas}
\subsection{3.1}{The double angle formulas}
\subsubsection{3.1.1}{Example}
\subsubsection{3.1.2}{A more complicated example}
\subsubsection{3.1.3}{Example without the double angle trick}

\section{5}{Integration by Parts}
\subsection{5.1}{The product rule and integration by parts}
\subsection{5.2}{An Example -- Integrating by parts once}
\subsection{5.3}{Another example}
\subsection{5.4}{Example -- Repeated Integration by Parts}
\subsection{5.5}{Another example of repeated integration by parts}
\subsection{5.6}{Example -- sometimes the factor $G'(x)$ is invisible}
\subsection{5.7}{An example where we get the original integral back}
\section{6}{Reduction Formulas}
\subsection{6.1}{First example of a reduction formula}
\subsection{6.2}{Reduction formula requiring two partial integrations}
\subsection{6.3}{A reduction formula where you have to solve for $I_n$}
\subsection{6.4}{A reduction formula that will come in handy later}

\section{8}{Partial Fraction Expansion}
\subsection{8.1}{Reduce to a proper rational function}
\subsection{8.2}{Example}
\subsection{8.3}{Partial Fraction Expansion: The Easy Case}
\subsection{8.4}{Previous Example continued}
\subsection{8.5}{Partial Fraction Expansion: The General Case}
\subsection{8.6}{Example}
\subsection{8.7}{A complicated example}
\subsection{8.8}{After the partial fraction decomposition}

\section{10}{Substitutions for integrals containing the expression $\sqrt {ax^2+bx+c}$}
\subsection{10.1}{Integrals involving $\sqrt {ax+b}$}
\subsubsection{10.1.1}{Example}
\subsubsection{10.1.2}{Another example}
\subsection{10.2}{Integrals containing $\sqrt {1-x^2}$}
\subsubsection{10.2.1}{Example}
\subsubsection{10.2.2}{Example: sometimes you don't have to do a trig substitution}
\subsection{10.3}{Integrals containing $\sqrt {a^2-x^2}$}
\subsubsection{10.3.1}{Example}
\subsection{10.4}{Integrals containing $\sqrt {x^2-a^2}$ or $\sqrt {a^2+x^2}$}
\subsubsection{10.4.1}{Example -- turn the integral $\int_2^4 \sqrt {x^2-4}{d}x$ into a trigonometric integral}
\section{11}{Rational substitution for integrals containing $\sqrt {x^2-a^2}$ or $\sqrt {a^2+x^2}$}
\subsection{11.1}{The functions $U(t)$ and $V(t)$}
\subsubsection{11.1.1}{Example \S \ref {sec:01area-under-hyperbola} again}
\subsubsection{11.1.2}{An example with $\sqrt {1+x^2}$}
\section{12}{Simplifying $\sqrt {ax^2+bx+c}$ by completing the square}
\subsection{12.1}{Example}
\subsection{12.2}{Example}
\subsection{12.3}{Example}

\section{14}{Chapter summary}

\chapter{II}{Proper and Improper Integrals}
\section{1}{Typical examples of improper integrals}
\subsection{1.1}{Integral on an unbounded interval}
\subsection{1.2}{Second example on an unbounded interval}
\subsection{1.3}{An improper integral on a finite interval}
\subsection{1.4}{A doubly improper integral}
\subsection{1.5}{Another doubly improper integral}
\section{2}{Summary: how to compute an improper integral}
\subsection{2.1}{How to compute an improper integral on an unbounded interval}
\subsection{2.2}{How to compute an improper integral of an unbounded function}
\subsection{2.3}{Doubly improper integrals}
\section{3}{More examples}
\subsection{3.1}{Area under an exponential}
\subsection{3.2}{Improper integrals involving $x^{-p}$}

\section{5}{Estimating improper integrals}
\subsubsection{}{Integral of a positive function}
\subsubsection{}{Comparison with easier integrals}
\subsubsection{}{Only the tail matters}
\subsection{5.1}{Improper integrals of positive functions}
\subsubsection{5.1.1}{Theorem}
\subsubsection{5.1.2}{Example - integral to infinity of the cosine}
\subsection{5.2}{Comparison Theorem for Improper Integrals}
\subsubsection{5.2.1}{Example}
\subsubsection{5.2.2}{Second example}
\subsection{5.3}{The Tail Theorem}
\subsubsection{5.3.1}{Example}
\subsubsection{5.3.2}{The area under the bell curve}

\chapter{III}{First order differential Equations}
\section{1}{What is a Differential Equation?}
\section{2}{Two basic examples}
\subsection{2.1}{Equations where the RHS does not contain $y$}
\subsection{2.2}{The exponential growth example}
\subsection{2.3}{Summary}
\section{3}{First Order Separable Equations}
\subsection{3.1}{Solution method for separable equations}
\subsubsection{}{Determining the constant}
\subsection{3.2}{A snag: }
\subsection{3.3}{Example}
\subsection{3.4}{Example: the snag in action}

\section{5}{First Order Linear Equations}
\subsection{5.1}{The Integrating Factor}
\subsection{5.2}{An example}
\subsubsection{}{Solution}

\section{7}{Direction Fields}
\section{8}{Euler's method}
\subsection{8.1}{The idea behind the method}
\subsection{8.2}{Setting up the computation}

\section{10}{Applications of Differential Equations}
\subsection{10.1}{Example: carbon dating}
\subsection{10.2}{Example: dating a leaky bucket}
\subsubsection{}{The snag appears again}
\subsection{10.3}{Heat transfer}
\subsection{10.4}{Mixing problems}
\subsubsection{}{Problem:}
\subsubsection{}{Solution:}

\chapter{IV}{Taylor's Formula}
\section{1}{Taylor Polynomials}
\subsection{1.1}{Definition}
\subsection{1.2}{Theorem}
\section{2}{Examples}
\subsection{2.1}{Taylor polynomials of order zero and one}
\subsection{2.2}{Example: Compute the Taylor polynomials of degree 0, 1 and 2 of $f (x)=e^x$ at $a=0$, and plot them}
\subsection{2.3}{Example: Find the Taylor polynomials of $f (x)=\sin x$}
\subsection{2.4}{Example: compute the Taylor polynomials of degree two and three of $f(x) = 1+x+x^2+x^3$ at $a=3$}
\subsubsection{}{Solution: }
\section{3}{Some special Taylor polynomials}

\section{5}{The Remainder Term}
\subsection{5.1}{Definition}
\subsection{5.2}{Example}
\subsection{5.3}{An unusual example, in which there {\bf is} a simple formula for $R_nf(x)$}
\subsection{5.4}{Another unusual, but important example where we can compute $R_nf(x)$}
\section{6}{Lagrange's Formula for the Remainder Term}
\subsection{6.1}{Theorem}
\subsection{6.2}{Estimate of remainder term}
\subsection{6.3}{How to compute $e$ in a few decimal places}
\subsection{6.4}{Error in the approximation $\sin x\approx x$}
\subsubsection{6.4.1}{Question: how big is the error in this approximation?}
\subsubsection{6.4.2}{Question: How small must we choose $x$ to be sure that the approximation $\sin x\approx x$ isn't off by more than $1$\% ?}

\section{8}{The limit as $x\to 0$, keeping $n$ fixed }
\subsection{8.1}{Little-oh}
\subsection{8.2}{Theorem}
\subsection{8.3}{Definition}
\subsection{8.4}{Example: prove one of these little-oh rules}
\subsection{8.5}{Can we see that $x^3=o (x^2)$ by looking at the graphs of these functions?}
\subsection{8.6}{Example: Little-oh arithmetic is a little funny}
\subsection{8.7}{Computations with Taylor polynomials}
\subsection{8.8}{Theorem}
\subsection{8.9}{How {\it NOT} to compute the Taylor polynomial of degree 12 of $f(x)=1/ (1+x^2)$}
\subsection{8.10}{A much easier approach to finding the Taylor polynomial of {\bf any} degree of $f(x)=1/ (1+x^2)$}
\subsection{8.11}{Example of multiplication of Taylor polynomials}
\subsection{8.12}{Taylor's formula and Fibonacci numbers}
\subsection{8.13}{More about the Fibonacci numbers}

\section{10}{Differentiating and Integrating Taylor polynomials}
\subsection{10.1}{Theorem}
\subsection{10.2}{Example: Taylor polynomial of $(1-x)^{-2}$}
\subsection{10.3}{Example: Taylor polynomials of $\arctan x$}

\section{12}{Proof of Theorem \ref {thm:fisg-to-order-n}}
\subsection{12.1}{Lemma}
\section{13}{Proof of Lagrange's formula for the remainder}
\chapter{V}{Sequences and Series}
\section{1}{Introduction}
\subsection{1.1}{A different point of view on Taylor expansions}
\subsection{1.2}{Some sums with infinitely many terms}
\section{2}{Sequences}
\subsection{2.1}{Examples of sequences}
\subsection{2.2}{Definition}
\subsection{2.3}{Example: $\displaystyle \lim_{n\to \infty } {1\over n}:=0$}
\subsection{2.4}{Example: $\displaystyle \lim_{n\to \infty } a^n=0$ if $|a|<1$}
\subsubsection{}{The case $|a|\geq 1$ (without proof)}
\subsection{2.5}{Theorem}
\subsection{2.6}{Sandwich theorem}
\subsection{2.7}{Theorem}
\subsection{2.8}{Example}
\subsection{2.9}{Limits of rational functions}
\subsection{2.10}{Example. Application of the Sandwich theorem. }
\subsubsection{}{Method 1: }
\subsubsection{}{Method 2: }
\subsection{2.11}{Example: factorial beats any exponential}

\section{4}{Series}
\subsection{4.1}{Definitions}
\subsection{4.2}{Examples}
\subsubsection{}{The geometric series}
\subsubsection{}{Telescoping series}
\subsection{4.3}{Properties of series}
\subsection{4.4}{Theorem}
\subsubsection{}{Rearranging terms in a series}
\section{5}{Convergence of Taylor Series}
\subsection{5.1}{The {\bf Geometric series} converges for $-1<x<1$}
\subsection{5.2}{Convergence of the exponential Taylor series}
\subsection{5.3}{The day that all Chemistry stood still}

\section{7}{Leibniz' formulas for $\ln 2$ and $\pi /4$}

\chapter{VI}{Vectors}
\section{1}{Introduction to vectors}
\subsection{1.1}{Definition}
\subsection{1.2}{Basic arithmetic of vectors}
\subsection{1.3}{Some GOOD examples. }
\subsection{1.4}{Two very, very BAD examples. }
\subsection{1.5}{Algebraic properties of vector addition and multiplication}
\subsubsection{}{Prove (\ref {eq:vector-addition-cmttve})}
\subsection{1.6}{Example}
\subsubsection{1.6.1}{Problem}
\subsubsection{1.6.2}{Problem}
\section{2}{Geometric description of vectors}
\subsection{2.1}{Definition}
\subsection{2.2}{Example}
\subsection{2.3}{Example}
\subsection{2.4}{Geometric interpretation of vector addition and multiplication}
\subsection{2.5}{Example}
\section{3}{Parametric equations for lines and planes}
\subsection{3.1}{Example}
\subsection{3.2}{Midpoint of a line segment. }
\section{4}{Vector Bases}
\subsection{4.1}{The Standard Basis Vectors}
\subsection{4.2}{A Basis of Vectors (in general)*}
\subsection{4.3}{Definition}
\subsection{4.4}{Definition}
\section{5}{Dot Product}
\subsection{5.1}{Definition}
\subsection{5.2}{Algebraic properties of the dot product}
\subsection{5.3}{Example}
\subsection{5.4}{The diagonals of a parallelogram}
\subsection{5.5}{Theorem}
\subsection{5.6}{The dot product and the angle between two vectors}
\subsection{5.7}{Theorem}
\subsection{5.8}{Orthogonal projection of one vector onto another}
\subsection{5.9}{Defining equations of lines}
\subsection{5.10}{Line through one point and perpendicular to another line}
\subsection{5.11}{Distance to a line}
\subsection{5.12}{Defining equation of a plane}
\subsection{5.13}{Example}
\subsection{5.14}{Example continued}
\subsection{5.15}{Where does the line through the points $B (2,0,0)$ and $C (0,1,2)$ intersect the plane ${\cal P}$ from example \ref {ex:defining-eqn-plane}?}
\section{6}{Cross Product}
\subsection{6.1}{Algebraic definition of the cross product}
\subsection{6.2}{Definition}
\subsection{6.3}{Example}
\subsection{6.4}{Example}
\subsection{6.5}{Algebraic properties of the cross product}
\subsection{6.6}{Ways to compute the cross product}
\subsection{6.7}{Example}
\subsection{6.8}{The triple product and determinants}
\subsection{6.9}{Definition}
\subsection{6.10}{Theorem}
\subsection{6.11}{Geometric description of the cross product}
\subsection{6.12}{Theorem}
\subsection{6.13}{Theorem}
\section{7}{A few applications of the cross product}
\subsection{7.1}{Area of a parallelogram}
\subsection{7.2}{Example}
\subsection{7.3}{Finding the normal to a plane}
\subsection{7.4}{Example}
\subsection{7.5}{Volume of a parallelepiped}
\section{8}{Notation}
\subsection{}{Common abuses of notation that should be avoided}





\vfill\eject\end
