%! To typeset this document use ''pdflatex free234''
\documentclass[12pt]{amsart}
\usepackage{amsmath}
\usepackage{amssymb}

\usepackage[landscape]{geometry}

\usepackage{graphicx,color}
\definecolor{darkgreen}{rgb}{0.0,0.5,0.0}
\usepackage{hyperref}
\hypersetup{colorlinks, 
            citecolor=black,
            filecolor=black,
            linkcolor=blue,
            urlcolor=darkgreen,
            bookmarksopen=true,
            pdftex}

\usepackage{url}

\setlength{\columnsep}{36pt}
\setcounter{tocdepth}{2}

\usepackage{eucal}


\input{free-macros234.tex}


\begin{document}

\title{Animations in GraphCalc.exe}
\maketitle

You can download the graphing calculator for windows at the following website
\begin{center}
  \url{http://www.graphcalc.com}
\end{center}
Follow the instructions on the ``download'' page of that site.

To plot an animation, or a moving graph of a function of two variables
$y=f(x, t)$ you go through these steps:
\begin{itemize}
\item Open the graphing calculator, and select the \textsf{graph 1} tab.
  You will see an $xy$ plane, and a list of ten equations you can enter.

\item On the line \textsf{y1} enter an equation (e.g.\ \verb|y=x^3-3*x|,
  which is how you have to enter $y=x^3-3x$) and mark the checkbox next to
  it.  The graph should appear.

\item To start an animation you have to enter the function of two variables
  $y=f(x, t)$. \emph{ The windows graphing calculator requires you to call the
  time variable $n$ instead of anything else.}  You could enter
  \verb|n^3-3*n+(3*n^2-3)*(x-n)| on the line for \textsf{y2}.

  Nothing happens.


\item To make the animation appear you now click on \textsf{2D Graph} in
  the toolbar at the top of the window, select \textsf{Analysis} and then
  select \textsf{N-Slider}.

  A window shows up with a slider and three boxes in which you can enter
  numbers.  If you entered the \textsf{y1} and \textsf{y2} from above, then
  you could now choose \verb|-2|, \verb|0.1|, and \verb|2| for the
  \textsf{Min}, \textsf{Step}, and \textsf{Max}.  Hit \textsf{animate} and
  the Movie should appear.
\end{itemize}

Note: how did I pick \textsf{y2}? Well, the tangent to the graph of
$y=f(x)$ at the point on the graph with $x=n$ has equation
\[
y = f(n) + f'(n) (x-n)
\]
as you know from math 221.  I just applied this to $f(x) = x^3-3x$.  You
could practice differentiating by choosing your own functions $f$.
\end{document}
